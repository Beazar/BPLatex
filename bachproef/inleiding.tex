%%=============================================================================
%% Inleiding
%%=============================================================================

\chapter{Inleiding}
\label{ch:inleiding}

Het ontwikkelen van webapplicaties is een complexe en tijdrovende zaak. Het gebruik van een framework is dan ook noodzakelijk om tijd en kosten te besparen. Vandaag de dag zijn er echter enorm veel frameworks beschikbaar, waardoor het moeilijker wordt om door de bomen het bos nog te zien. De keuze voor een passend framework is dus geen gemakkelijke zaak. 

Elk framework heeft sterktes en zwaktes en zal excelleren in verschillende omstandigheden. Bijgevolg kunnen we nooit eenduidig concluderen dat een bepaald framework beter is dan een ander framework. In deze scriptie zullen we een vergelijking maken tussen Vaadin en Angular binnen een bepaalde context. Aan de hand hiervan kunnen we concluderen welk framework het meest performant is binnen deze context. 

De keuze voor de twee bovengenoemde frameworks valt te verklaren door het feit dat ze beiden component-gebaseerde frameworks zijn voor single page applicaties. Bovendien kunnen beide frameworks gebruik maken van de JavaScript taal. 

\iffalse
De inleiding moet de lezer net genoeg informatie verschaffen om het onderwerp te begrijpen en in te zien waarom de onderzoeksvraag de moeite waard is om te onderzoeken. In de inleiding ga je literatuurverwijzingen beperken, zodat de tekst vlot leesbaar blijft. Je kan de inleiding verder onderverdelen in secties als dit de tekst verduidelijkt. Zaken die aan bod kunnen komen in de inleiding~\autocite{Pollefliet2011}:

\begin{itemize}
  \item context, achtergrond
  \item afbakenen van het onderwerp
  \item verantwoording van het onderwerp, methodologie
  \item probleemstelling
  \item onderzoeksdoelstelling
  \item onderzoeksvraag
  \item \ldots
\end{itemize}
\fi

\section{Context}
\label{sec:context}
Angular is een open-source framework dat al ruime tijd beschikbaar is. De eerste  release vond al plaats in 2010, onder de naam "AngularJS". 
In 2014 werd een nieuwere versie ontwikkeld die de naam "Angular 2" kreeg. Dit leidde echter tot verwarring bij de gebruikers waardoor het team besloot om te verwijzen naar de eerste versie als "AngularJS"  en naar de daarop volgende versies te verwijzen als "Angular". 

Angular 7 is de meest recente versie van Angular werd in oktober 2018 op de markt gebracht. Deze versie zal dan ook besproken in deze scriptie. 

Net zoals Angular is ook Vaadin een open-source framework. Het is echter een nieuwer  framework aangezien de eerste versie van Vaadin pas in 2017 op de markt kwam. 

Vaadin 12 is de meest recente versie die momenteel beschikbaar is. Deze werd in december 2018 uitgebracht. Vaadin 12 zal dan ook besproken worden in deze scriptie. 
De volgende versie van Vaadin wordt uitgebracht in maart 2019.

\section{Probleemstelling}
\label{sec:probleemstelling}
\iffalse
Uit je probleemstelling moet duidelijk zijn dat je onderzoek een meerwaarde heeft voor een concrete doelgroep. De doelgroep moet goed gedefinieerd en afgelijnd zijn. Doelgroepen als ``bedrijven,'' ``KMO's,'' systeembeheerders, enz.~zijn nog te vaag. Als je een lijstje kan maken van de personen/organisaties die een meerwaarde zullen vinden in deze bachelorproef (dit is eigenlijk je steekproefkader), dan is dat een indicatie dat de doelgroep goed gedefinieerd is. Dit kan een enkel bedrijf zijn of zelfs één persoon (je co-promotor/opdrachtgever).
\fi
Het kiezen voor een bepaald framework is een zeer complexe zaak doordat er enorm veel frameworks beschikbaar zijn. Voor webdevelopers is het onmogelijk om elk framework te vergelijken. 
Bovendien zal bij elke situatie een ander framework als beste presteren. 
Deze scriptie zal trachten ervoor te zorgen dat de keuze tussen Vaadin en Angular  vereenvoudigd wordt. 

\section{Onderzoeksvraag}
\label{sec:onderzoeksvraag}
\iffalse
Wees zo concreet mogelijk bij het formuleren van je onderzoeksvraag. Een onderzoeksvraag is trouwens iets waar nog niemand op dit moment een antwoord heeft (voor zover je kan nagaan). Het opzoeken van bestaande informatie (bv. ``welke tools bestaan er voor deze toepassing?'') is dus geen onderzoeksvraag. Je kan de onderzoeksvraag verder specifiëren in deelvragen. Bv.~als je onderzoek gaat over performantiemetingen, dan 
\fi
\subsection{Hoofdonderzoeksvraag}
Zoals eerder vermeld in de inleiding zal deze scriptie zich focussen op het vergelijken van de mogelijkheden van Vaadin en Angular. De vergelijking zal gebeuren aan de hand van volgende onderzoeksvraag:
\begin{itemize}
	\item in welke situatie kiest men best voor het Vaadin framework en wanneer opteert men best voor Angular?
\end{itemize}
Het antwoord op deze vraag zal gegeven worden in de conclusie in hoofdstuk \ref*{ch:conclusie}.

\subsection{Deelonderzoeksvragen}
Deze hoofdonderzoeksvraag wordt opgesplitst in meerder deelonderzoeksvragen die hieronder worden opgelijst:
\begin{itemize}
	\item Wat hebben beide frameworks gemeenschappelijk met elkaar?
	\item Wat zijn de belangrijkste verschillen tussen beide frameworks?
	\item Welk framework heeft de beste ondersteuning voor mobile?
	\item Welk framework heeft de beste ondersteuning voor testing?
	\item Welk framework heeft de hoogste performantie?
	\item Wat vinden webontwikkelaars van beide frameworks?
\end{itemize}

\section{Onderzoeksdoelstelling}
\label{sec:onderzoeksdoelstelling}

Het hoofddoel van dit onderzoek is om de keuze van webdevelopers tussen twee component-gebaseerde frameworks, Vaadin en Angular te vereenvoudigen. 
Dit zal gebeuren aan de hand van verschillende criteria zoals de geschiktheid voor mobile, performantie en de beschikbare bibliotheken voor tests. Verder zal dit onderzoek ook handelen over de gelijkenissen en verschillen tussen de twee voorgenoemde frameworks. 
Ten slotte zullen ook de meningen van webontwikkelaars weergegeven worden.
Dit onderzoek zal gebeuren aan de hand van twee zaken:
\begin{itemize}
	\item Een uitgebreide literatuurstudie
	\item Een single page applicatie opgebouwd aan de hand van beide frameworks
	
\end{itemize}

\section{Opzet van deze bachelorproef}
\label{sec:opzet-bachelorproef}

% Het is gebruikelijk aan het einde van de inleiding een overzicht te
% geven van de opbouw van de rest van de tekst. Deze sectie bevat al een aanzet
% die je kan aanvullen/aanpassen in functie van je eigen tekst.

De rest van deze bachelorproef is als volgt opgebouwd:

In Hoofdstuk~\ref{ch:stand-van-zaken} wordt een overzicht gegeven van de stand van zaken binnen het onderzoeksdomein, op basis van een literatuurstudie.

In Hoofdstuk~\ref{ch:methodologie} wordt de methodologie toegelicht en worden criteria opgesteld aan de hand van welke de frameworks vergeleken zullen worden.

In Hoofdstuk~\ref{ch:angular} worden beide frameworks dan vergeleken aan de hand van de vooropgestelde criteria.

% TODO: Vul hier aan voor je eigen hoofstukken, één of twee zinnen per hoofdstuk

In Hoofdstuk~\ref{ch:conclusie}, tenslotte, wordt de conclusie gegeven en een antwoord geformuleerd op de onderzoeksvragen. Daarbij wordt ook een aanzet gegeven voor toekomstig onderzoek binnen dit domein.

