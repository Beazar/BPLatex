%%=============================================================================
%% Voorwoord
%%=============================================================================

\chapter*{Woord vooraf}
\label{ch:voorwoord}
Voor u ligt de scriptie 'Vaadin versus Angular, een vergelijking'. Deze scriptie is geschreven in het kader van het afwerken van mijn studie Toegepaste Informatica aan de hogeschool Hogent campus Aalst. Van oktober 2018 tot en met mei 2019 ben ik bezig geweest met het onderzoeken en schrijven van deze scriptie. 

De keuze voor Angular kwam er door persoonlijke interesse voor dit framework. In het tweede jaar van mijn opleiding kwam ik voor de eerste keer in aanraking met Angular. Dit gebeurde tijdens het vak 'Webapplicaties IV'. Sindsdien is de interesse in Angular steeds blijven groeien, mede doordat ik verschillende projecten heb gemaakt aan de hand van dit framework. Daarenboven kwam ik ook in aanraking met Angular op mijn stageplek Realdolmen. 

De keuze voor Vaadin kwam er op aanraden van mijn promotor Chantal Teerlinck. Zij wou graag meer te weten komen over dit component gebaseerde framework. Aangezien er amper vergelijkende studies rond Vaadin zijn uitgevoerd bleek dit een goed onderzoek voor een scriptie.

Tijdens dit onderzoek stonden mijn promotor, Chantal Teerlinck, en mijn co-promotors vanuit mijn stageplek, Geerard Ponnet en Reinout Claeys, altijd voor mij klaar. Zij waren steeds paraat om mijn vragen te beantwoorden en gaven mij de nodige richtlijnen om deze scriptie te voltooien.

Bij deze wil ik graag mijn begeleiders bedanken voor hun raad en ondersteuning tijdens het het schrijven van deze scriptie. Zonder hun medewerking had ik dit onderzoek nooit kunnen voltooien.
Verder wil ik graag mijn vrienden en familie bedanken die de tijd hebben genomen om mijn scriptie na te lezen. 
Ten slotte wil ik mijn ouders en vriendin bedanken. Zij hebben me moreel ondersteund tijdens het schrijfproces en hielpen deze scriptie tot een goed einde te brengen.

Ik wens u veel leesplezier toe.

Sander Beazar

Aalst, 29 mei 2019




%% TODO:
%% Het voorwoord is het enige deel van de bachelorproef waar je vanuit je
%% eigen standpunt (``ik-vorm'') mag schrijven. Je kan hier bv. motiveren
%% waarom jij het onderwerp wil bespreken.
%% Vergeet ook niet te bedanken wie je geholpen/gesteund/... heeft

