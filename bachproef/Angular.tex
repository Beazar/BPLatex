\chapter{Vergelijking Angular en Vaadin}
\label{ch:angular}
\section{Mobile}
In de afgelopen jaren is de ontwikkeling van mobiele applicaties een "booming business" geworden. Dankzij mobiele ondersteuning blijft een applicatie altijd en overal bereikbaar, het zorgt voor een direct marketingkanaal, een grotere merkherkenning, verhoogt het inkomen... Alle voorgaande redenen en nog vele andere bepalen dat mobiele ondersteuning voor een framework belangrijk is.

\begin{figure}[H]
	\centering
	\includegraphics[width=0.6\linewidth]{Native-Web-progressive}
	\caption{Native App vs Responsive App vs Progressive App \autocite{Solis2018}}
	\label{fig:native-responsive-web}
\end{figure}
Bij het ontwikkelen van een mobiele applicatie heeft de ontwikkelaar drie opties(zie figuur \ref{fig:native-responsive-web}):
\begin{itemize}
	\item Responsive webapplicaties: De webapplicatie zal zich aanpassen aan het device waarop de applicatie wordt gedraaid. Deze aanpak is kostefficiënt omdat er maar een applicatie moet gebouwd worden voor zowel desktop als mobiele gebruikers. Verder moet is er geen download nodig, aangezien de webapplicaties gewoon via de browser kan opgeroepen worden.
	\item Native applicatie: De applicatie wordt specifiek gebouwd voor native platformen zoals iOS of Android. De voordelen van native applicaties is dat ze een betere performantie hebben, offline beschikbaar zijn en push notificaties kunnen sturen. Anderzijds moet elke applicatie gebouwd en getest worden voor elk ondersteund platform.
	\item Progressive webapplicaties(PWA):  Progressive webapplicaties zijn websites gebouwd met webtechnologie die zich gedragen als een app. Ze zijn de opvolgers van de 'Hybrid Applicatie' maar in tegenstelling tot de Hybrid Applicatie vereisen ze geen download. Net als de native applicaties kunnen ze geraadpleegd worden zonder internetverbinding en kunnen ze push notificaties uitsturen. 
\end{itemize}

Vaadin en Angular hebben zeer verschillende aanpakken met betrekking tot mobiele ondersteuning. Waar de focus van Vaadin op het web ligt, zorgt Angular voor herbruikbaarheid van delen van de applicatie om mobiele applicaties te bouwen.
\subsection{Angular}
De ontwikkelaar heeft verschillende keuzes wanneer hij een mobiele applicatie wil ontwikkelen met Angular. Hij kan een zowel een responsive webapplicatie als een progressive webapplicatie maken. Hij kan zelfs een native applicatie maken wanneer hij de template maakt met NativeScript in plaats van HTML. 

Het responsief maken van de applicatie gebeurt aan de hand van CSS media query's. Deze media query's staan de ontwikkelaar toe om verschillende regels voor de layout toe te voegen afhankelijk van de grootte van het scherm van het device. Verder kunnen de ontwikkelaars ook listeners toevoegen in de TypeScript code die reageren veranderingen in de grootte van het scherm. Zo kunnen bepaalde componenten veranderd worden als de grootte van het scherm verandert. 

\subsection{Vaadin}
Wanneer de ontwikkelaar ervoor kiest om de mobiele applicatie te bouwen in Vaadin heeft hij twee keuzes: een responsive webapplicatie of een progressive webapplicatie. 
Voor het ontwikkelen van een progressive webapplicatie zijn er vanuit een technisch standpunt twee dingen noodzakelijk: een  Web App Manifest file en een ServiceWorker JavaScript file. De manifest file zorgt voor metadata zoals kleuren en iconen voor operating system integratie, de ServiceWorker zorgt voor caching en push notificaties.
Terwijl de ServiceWorker en de manifest file in Angular automatisch worden aangemaakt, moeten deze bij Vaadin manueel aangemaakt worden. Verder zal de applicatie ook niet offline beschikbaar zijn aangezien de application state wordt bijgehouden door de server.
Dit leidt ertoe dat het opzetten van een progressive webapplicatie in Vaadin moeilijker is en langer duurt.  
Vanaf de volgende versie van Vaadin (Vaadin 13) zou de ServiceWorker en de manifest file echter wel aan het framework toegevoegd worden. 
\subsection{Conclusie}
Angular zorgt voor een grotere flexibiliteit dan Vaadin bij het ontwikkelen van mobiele applicaties aangezien de developer ook kan opteren voor een native applicatie. Het ontwikkelen van een progressive webapplicatie verloopt bij Angular ook sneller en eenvoudiger dan bij Vaadin. Bovendien zal de Angular progressive webapplicatie ook offline beschikbaar zijn, waar dit voor een Vaadin progressive webapplicatie niet het geval is. Bovenstaande redenen zorgen ervoor dat Angular beter scoort dan Vaadin op het onderdeel mobile. \\
Angular krijgt voor deze requirement een score van 4, Vaadin een score van 2.

\section{Browsercompatibiliteit}
Browsercompatibiliteit is de manier waarop een webpagina er anders uitziet in verschillende webbrowsers. Verschillende browsers lezen de code van een website op een verschillende manier. Hierdoor zal Chrome een website soms anders weergeven dan FireFox of Internet Explorer. Voor een website is het belangrijk dat hij compatibel is met verschillende browsers omdat niet alle gebruikers dezelfde browser gebruiken. 
De browsercompatibiliteit van beide frameworks is weergegeven in tabel  \ref{table:browsercompatibiliteit}.

\subsection{Conclusie}
Beide frameworks zijn compatibel met alle populaire browsers. Daarom krijgen zowel Vaadin als Angular de maximumscore van 5 op deze requirement.


\begin{table}[H]
	\begin{tabular}{|l|l|l|l|l|l|}
		\hline
		\textbf{Framework} & \textbf{Chrome} & \textbf{FireFox} & \textbf{Internet Explorer} & \textbf{Edge}     & \textbf{Safari}   \\ \hline
		\textbf{Angular}   & Laatste versie  & Laatste versie   & 11,10,9                    & 2 laatste versies & 2 laatste versies \\ \hline
		\textbf{Vaadin}    & Laatste versie  & Laatste versie   & 11,10                      & Laatste versie    & Geen info         \\ \hline
	\end{tabular}
\caption{Compatibiliteit met browsers}
\label{table:browsercompatibiliteit}
\end{table}


\section{Data binding}
Het implementeren van data binding is belangrijk voor frameworks aangezien het heel wat tijd kan besparen. Er zijn verschillende soorten data binding die we voor elk framework zullen bespreken.
\\  \\
\textbf{one-way data binding} \hspace{1cm} Een aanpassing in het model zal automatisch de views/templates aanpassen.
\\
\textbf{two-way data binding} \hspace{1cm} Het veranderen van de view resulteert in een verandering in het model en een verandering in het model resulteert in een aangepaste view. 
\\
\textbf{three-way data binding} \hspace{1cm} De data wordt gesynchroniseerd met een  remote storage. 
\\
\textbf{four-way data binding} \hspace{1cm} De data wordt gesynchroniseerd met een  lokale database, die op zijn beurt weer gesynchroniseerd wordt met een remote storage.

\subsection{Angular}

\subsubsection{one-way data binding}
\begin{figure}[H]
\begin{lstlisting}
<div>{{contact.voornaam}}</div>
\end{lstlisting}
\caption{One-way data binding Angular}
\end{figure}
\subsubsection{two-way data binding}

\begin{figure}[H]
\begin{lstlisting}
<input [(ngModel)]="contact.afbeelding"\>
\end{lstlisting}
\caption{Two-way data binding Angular}
\end{figure}
\subsubsection{three-way data binding}
Geen standaard implementatie, de ontwikkelaar moet dit zelf schrijven.

\subsubsection{four-way data binding}
Geen standaard implementatie, de ontwikkelaar moet dit zelf schrijven.
\subsection{Vaadin}
\subsubsection{one-way data binding}

\begin{figure}[H]
\begin{lstlisting}
<div>[[voornaam]]</div>
\end{lstlisting}
\caption{One-way data binding Vaadin}
\end{figure}
\subsubsection{two-way data binding}
\begin{figure}[H]
\begin{lstlisting}
<input value="{{voornaam::input}}">
\end{lstlisting}
\caption{Two-way data binding Vaadin}
\end{figure}

\subsubsection{three-way data binding}
Geen standaard implementatie, de ontwikkelaar moet dit zelf schrijven.
\subsubsection{four-way data binding}
Geen standaard implementatie, de ontwikkelaar moet dit zelf schrijven.

\subsubsection{Conclusie}
Beide frameworks hebben een eenvoudige implementatie van zowel one-way als two-way data binding. Beide frameworks kunnen zich nog verbeteren door het implementeren van three-way data binding en four-way data binding te vereenvoudigen. Hierdoor krijgen beide frameworks een score van 3. 

\section{Moeilijkheidsgraad}
Het aanleren van een framework is niet altijd eenvoudig. Hoe sneller een developer een framework onder de knie kan krijgen hoe beter. 
Angular is vrij moeilijk om aan te leren: de programmeur moet namelijk drie verschillende talen leren.
Eerst en vooral moet hij TypeScript leren. Hiervoor heeft hij een voorkennis nodig van JavaScript. Daarenboven moet hij ook een uitgebreide kennis hebben van HTML en CSS voor het maken en stylen van de templates.
\\
In tegenstelling tot Angular is Vaadin veel eenvoudiger om te leren. De programmeur moet enkel Java kennen, kennis van HTML en CSS is een pluspunt maar niet noodzakelijk.
Doordat er enkel kennis van Java nodig is, is Vaadin een framework dat gemakkelijk aangeleerd kan worden door programmeurs die geen ervaring hebben in de web-development sector.

\subsection{Conclusie}
Angular is een vrij moeilijk framework om aan te leren doordat de programmeur drie talen moet kennen. Vaadin is makkelijker onder de knie te krijgen aangezien applicaties geschreven kunnen worden met kennis van één taal, namelijk Java. 
Voor deze reden krijgt Vaadin een score van 4 en Angular een score van 2.


\section{Open source}
Open source software is software waarvan de broncode is gepubliceerd en vrij te gebruiken is voor het publiek. Iedereen kan op die manier kopiëren, aanpassen en verspreiden zonder kosten te betalen voor auteursrechten.
Bij de ontwikkeling van een webapplicatie of een mobiele applicatie is het dus voordelig wanneer de frameworks gratis gebruikt kunnen worden. 

\subsection{Angular}
Angular wordt aangeboden onder de MIT-licentie.
De MIT-licentie is een softwarelicentie voor opensourcesoftware. Deze MIT-licentie bijna alles toe. De enige voorwaarde is dat het copyright statement in alle kopieën moet blijven staan.
de broncode van Angular is beschikbaar op: 
\\
\url{https://github.com/angular/angular} \\ \\
Verder zijn er ook heel veel voorgemaakte componenten waarvan de broncode beschikbaar is. Angular heeft zelf ook een library met componenten die ze beschikbaar stellen onder de naam Material Design. Net als het framework worden deze componenten aangeboden onder de MIT-licentie. De broncode van deze library kan gevonden worden op:
\\
\url{https://github.com/angular/components}

\subsection{Vaadin}
Vaadin Framework is beschikbaar onder de Apache License, version 2.0.
De Apache-licentie is een softwarelicentie voor vrije software uitgebracht door de Apache Software Foundation. De licentie staat toe dat deze software aangepast en/of opnieuw verspreid mag worden op voorwaarde dat een kopie van de Apache License wordt meegeleverd en de eerdere copyrightvermeldingen behouden blijven.
\\ De documentatie van het framework staat onder een andere licentie, namelijk de Creative Commons CC-BY-ND 2.0. Dit houdt in dat licentienemers  het werk mogen  kopiëren, verspreiden, tentoonstellen en uitvoeren en afgeleide werken maken wanneer zij de auteur of licentieverstrekker op een door deze gespecificeerde manier vermelden. Dit mag enkel gebeuren in ongewijzigde vorm.
\\De broncode van het framework kan gevonden worden op: 
\\ \url{https://github.com/vaadin/flow} \\ \\

Vaadin voorziet ook open source componenten onder de naam Vaadin core components. Deze worden net als het framework aangeboden onder de Apache License, version 2.0.
\\ De broncode van deze components is beschikbaar op 
\\ \url{https://github.com/vaadin/vaadin-core }

\subsection{Conclusie}
De MIT-licentie van Angular is kort en zeer eenvoudig. Alles mag aangepast worden aan de broncode op voorwaarde dat de originale licentie aanwezig blijft. 
De Apache License, version 2.0 waaronder Vaadin valt is net als de MIT-licentie een licentie dat heel veel toelaat. Het enige nadeel ten opzichte van de MIT-licentie is dat wanneer iets aangepast wordt aan de Apache-licensed code dit ook steeds moet aangegeven worden.
Aangezien dit maar een heel klein verschil betreft en beide frameworks in essentie volledig open source zijn, krijgen ze beiden de maximumscore van 5 op dit onderdeel.

\section{Testing}


\section{Populariteit}
De populariteit is een belangrijk criterium bij het beoordelen van een framework. Een framework met veel gebruikers heeft een grote community, waardoor het vinden van oplossingen bij problemen makkelijker verloopt dan bij een framework met een kleine community. Daarenboven bepaalt het aantal gebruikers ook de stabiliteit van een framework. Ten slotte bepaalt de populariteit ook het aantal library's die door de community ter beschikking gesteld worden.
\subsection{Stack Overflow}
Stack Overflow is een platform dat als doel heeft om programmeurs te voorzien van een goede Q\&A website. Maandelijks heeft Stack Overflow meer dan 50 miljoen bezoekers waardoor het de grootste online developer community genoemd mag worden. 
\subsubsection{Angular}
Zie tabel \ref{table:angularstackoverflow}(raadpleging op 8/04/2019) op pagina \pageref{table:angularstackoverflow}. \\ \\
Angular heeft een heel groot aantal aangemaakte issues wat erop wijst dat Angular een zeer populair framework is. Het percentage opgeloste issues ligt rond de 60 percent, wat een redelijk goede score is.
\begin{table}[H]
	\begin{tabular}{|l|l|l|l|}
		\hline
		Naam van de tag   & \textbf{Aantal aangemaakte issues} & \textbf{Aantal onopgeloste issues} & \textbf{\% opgeloste issues} \\ \hline
		\textbf{Angular}  & 161.326                            & 65.823                             & 59,20\%                              \\ \hline
		\textbf{Angular5} & 2.827                              & 1.193                              & 57,80\%                              \\ \hline
		\textbf{Angular6} & 2.763                              & 1.181                              & 57,26\%                              \\ \hline
		\textbf{Angular7} & 723                                & 320                                & 55,74\%                              \\ \hline
		\textbf{Totaal}   & \textbf{167.639}                   & \textbf{68.517}                    & \textbf{59.13\%}                     \\ \hline
	\end{tabular}
\caption{Populariteit van Angular op Stack Overflow}
\label{table:angularstackoverflow}
\end{table}
\subsubsection{Vaadin}
Zie tabel \ref{table:vaadinstackoverflow}(raadpleging op 8/04/2019) op pagina \pageref{table:vaadinstackoverflow}. \\ \\
Het aantal aangemaakte issues van Vaadin ligt laag. Dit wijst erop dat Vaadin niet veel gebruikers heeft. Het percentage opgeloste issues ligt echter zeer hoog, namelijk rond de 84 percent. Ondanks de kleinere omvang van community is de kans op een oplossing van een issue dus toch groot. \\

\begin{table}[H]
	\begin{tabular}{|l|l|l|l|}
		\hline
		Naam van de tag   & \textbf{Aantal aangemaakte issues} & \textbf{Aantal onopgeloste issues} & \textbf{\% opgeloste issues} \\ \hline
		\textbf{Vaadin}  & 4.588                            & 691                             & 84,94\%                              \\ \hline
		\textbf{Vaadin6} & 9                              & 1                              & 88,89\%                              \\ \hline
		\textbf{Vaadin7} & 386                              & 85                              & 77,98\%                              \\ \hline
		\textbf{Vaadin8} & 166                                & 49                                & 70,48\%                              \\ \hline
		\textbf{Vaadin10} & 36                                & 12                                & 66,66\%                              \\ \hline
		\textbf{Totaal}   & \textbf{5.185}                   & \textbf{838}                    & \textbf{83.84\%}                     \\ \hline
	\end{tabular}
	\caption{Populariteit van Vaadin op Stack Overflow}
	\label{table:vaadinstackoverflow}
\end{table}

\subsection{Reddit}
Reddit is een social sharing website waarbij gebruikers zelf de content aanleveren. Dit kunnen zowel artikels als afbeeldingen of video's zijn. Reddit bestaat uit verschillende subreddits die handelen over bepaalde onderwerpen. Het aantal subscribers van deze subreddits kunnen een indicator vormen voor de populariteit van het onderwerp.\\

Uit de gegevens van tabel \ref{table:reddit} kunnen we afleiden dat Angular veel populairder is dan Vaadin. Met meer dan 200 keer meer subscribers is het verschil zeer duidelijk. Bovendien is de subreddit r/vaadin ook veel minder actief dan r/angular.
\begin{table}[H]
	\begin{tabular}{|l|l|l|}
		\hline
		\textbf{Subreddit}  & \textbf{Aantal subscribers} & \textbf{Aantal posts in de laatste 30 dagen} \\ \hline
		\textbf{r/angular2} & 31.468                      & 79                                           \\ \hline
		\textbf{r/vaadin}   & 147                         & 0                                            \\ \hline
	\end{tabular}
	\caption{Populariteit van de frameworks op Reddit}
\label{table:reddit}
\end{table}

\subsection{Stack Overflow Developer Survey}
Elk jaar voert Stack Overflow zijn Developer Survey \autocite{DeveloperSurvey2019} uit, waarbij duizenden developers vertellen over wat ze leren, welke tools en frameworks ze gebruiken en welke job ze uitvoeren. De resultaten hiervan kunnen gebruikt worden om de populariteit te bepalen van deze onderdelen. Door de resultaten de afgelopen jaren te gebruiken kan men bovendien een inzicht krijgen in de evolutie van de populariteit.


\subsubsection{Resultaten 2017}
In 2017 was Angular het tweede meest gebruikte framework, met maar liefst 44 percent van de respondenten die reeds het framework gebruikt hadden. Bovendien vonden meer dan de helft van de gebruikers het een leuk framework om te gebruiken.

Vaadin werd niet genoemd in deze enquête wat erop wijst dat het in 2017 geen populair framework was.

\begin{figure}[H]
	\centering
	\includegraphics[width=0.6\linewidth]{2017Frameworks}
	\caption{Frameworks, Libraries, and Other Technologies 2017 \autocite{DeveloperSurvey2017}}
	\label{fig:Frameworks, Libraries, and Other Technologies 2017}
\end{figure}

\begin{figure}[H]
	\centering
	\includegraphics[width=0.6\linewidth]{2017LovedFrameworks}
	\caption{Most Loved Frameworks, Libraries and Other Technologies 2017 \autocite{DeveloperSurvey2017}}
	\label{fig:Most Loved Frameworks, Libraries and Other Technologies 2017}
\end{figure}

\subsubsection{Resultaten 2018}
In 2018 was Angular nog steeds het tweede meest gebruikte framework, maar het aantal respondenten die er al gebruik van gemaakt hadden daalde naar 37 percent. Nog steeds vonden meer dan de helft van de gebruikers het een leuk framework om te gebruiken.

Vaadin werd niet genoemd in deze enquête wat erop wijst dat het in 2018 geen populair framework was.

\begin{figure}[H]
	\centering
	\includegraphics[width=0.6\linewidth]{2018Frameworks}
	\caption{Frameworks, Libraries, and Other Technologies 2018 \autocite{DeveloperSurvey2018}}
	\label{fig:Frameworks, Libraries, and Other Technologies 2018}
\end{figure}

\begin{figure}[H]
	\centering
	\includegraphics[width=0.6\linewidth]{2018LovedFrameworks}
	\caption{Most Loved Frameworks, Libraries and Other Technologies 2018 \autocite{DeveloperSurvey2018}}
	\label{fig:Most Loved Frameworks, Libraries and Other Technologies 2018}
\end{figure}

\subsubsection{Resultaten 2019}
De daling van de populariteit van Angular zette zich ook in 2019 verder. Het zakte van de tweede naar de derde plaats van populairste framework. Het aantal respondenten die gebruik maakten van het framework zakte naar 31 percent. Nochtans steeg het percentage van ontwikkelaars die graag gebruik maken naar 57 percent.

Vaadin werd niet genoemd in deze enquête wat erop wijst dat het in 2019 geen populair framework was.

\begin{figure}[H]
	\centering
	\includegraphics[width=0.6\linewidth]{2019Frameworks}
	\caption{Frameworks, Libraries, and Other Technologies 2019 \autocite{DeveloperSurvey2019}}
	\label{fig:Frameworks, Libraries, and Other Technologies 2019}
\end{figure}

\begin{figure}[H]
	\centering
	\includegraphics[width=0.6\linewidth]{2019LovedFrameworks}
	\caption{Most Loved Frameworks, Libraries and Other Technologies 2019 \autocite{DeveloperSurvey2019}}
	\label{fig:Most Loved Frameworks, Libraries and Other Technologies 2019}
\end{figure}

\subsection{Google Trends}

Google Trends is een dienst van Google die via grafieken inzicht geeft wanneer en hoe vaak op een bepaald woord is gezocht met Google. \\ \\
De cijfers in figuur \ref{fig:Google Trends zoektermen} geven de zoekinteresse aan ten opzichte van het hoogste punt in het diagram voor de betreffende regio en periode. Een waarde van 100 is de piekpopulariteit voor die term. Een waarde van 50 betekent dat de term half zo populair is. Een score van 0 betekent dat er onvoldoende gegevens beschikbaar zijn voor deze term. De resultaten van de zoekterm "Angular" worden weergegeven door een rode lijn, de resultaten van de zoekterm "Vaadin" door een blauwe lijn.

Er kan geconcludeerd worden dat er wereldwijd veel vaker gezocht wordt naar het sleutelwoord "Angular" dan naar "Vaadin". 

\begin{figure}[H]
	\centering
	\includegraphics[width=0.6\linewidth]{GoogleTrendsTimeLine}
	\caption{Zoektermen Vaadin versus Angular \autocite{GoogleTrends2019}}
	\label{fig:Google Trends zoektermen}
\end{figure}

\subsection{Github}
GitHub is een populaire website waarop software kan geplaatst worden. GitHub is gebouwd rond het Git-versiebeheersysteem, waardoor GitHub alle mogelijkheden van Git en eigen toevoegingen aanbiedt. Het aantal watchers, stars en forks van een framework op Github is een indicator voor de populariteit van het framework. In tabel \ref{table:github} wordt nogmaals duidelijk dat Angular veel populairder is dan Vaadin.

\begin{table}[H]
	\begin{tabular}{|l|l|l|l|}
		\hline
		\textbf{Github}  & \textbf{Aantal watchers} & \textbf{Aantal stars} & \textbf{Aantal forks} \\ \hline
		\textbf{Angular} & 3284                     & 47.296                & 12.612                \\ \hline
		\textbf{Vaadin}  & 57                       & 184                   & 55                    \\ \hline
	\end{tabular}
	\caption{Populariteit van de frameworks op Github}
\label{table:github}
\end{table}

\subsection{Conclusie}
Angular is duidelijker veel populairder dan Vaadin, waardoor het framework stabieler is en de gebruiker kan rekenen op meer ondersteuning vanuit de community. Angular behoort tot de meest populaire frameworks, terwijl Vaadin niet populair is. De enige factor waar Vaadin beter op scoort dan Angular is het percentage opgeloste issues op Stack Overflow. 

Door de hierboven genoemde redenen krijgt Angular voor dit requirement een score van 5 en Vaadin een score van 2.



 