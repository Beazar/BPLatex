%%=============================================================================
%% Samenvatting
%%=============================================================================

% TODO: De "abstract" of samenvatting is een kernachtige (~ 1 blz. voor een
% thesis) synthese van het document.
%
% Deze aspecten moeten zeker aan bod komen:
% - Context: waarom is dit werk belangrijk?
% - Nood: waarom moest dit onderzocht worden?
% - Taak: wat heb je precies gedaan?
% - Object: wat staat in dit document geschreven?
% - Resultaat: wat was het resultaat?
% - Conclusie: wat is/zijn de belangrijkste conclusie(s)?
% - Perspectief: blijven er nog vragen open die in de toekomst nog kunnen
%    onderzocht worden? Wat is een mogelijk vervolg voor jouw onderzoek?
%
% LET OP! Een samenvatting is GEEN voorwoord!

%%---------- Nederlandse samenvatting -----------------------------------------
%
% TODO: Als je je bachelorproef in het Engels schrijft, moet je eerst een
% Nederlandse samenvatting invoegen. Haal daarvoor onderstaande code uit
% commentaar.
% Wie zijn bachelorproef in het Nederlands schrijft, kan dit negeren, de inhoud
% wordt niet in het document ingevoegd.

\IfLanguageName{english}{%
\selectlanguage{dutch}
\chapter*{Samenvatting}
\selectlanguage{english}
}{}

%%---------- Samenvatting -----------------------------------------------------
% De samenvatting in de hoofdtaal van het document

\chapter*{\IfLanguageName{dutch}{Samenvatting}{Abstract}}

De keuze voor een gepast framework bij het ontwikkelen van een web applicatie is niet eenvoudig. Er zijn enorm veel verschillende frameworks beschikbaar, die elk excelleren in verschillende domeinen. Deze scriptie tracht de keuze van de lezer tussen twee component gebaseerde frameworks, Angular en Vaadin, te vereenvoudigen. Dit gebeurt aan de hand van de onderzoeksvraag 'In welke situatie kiest men best voor het Vaadin framework en wanneer opteert men best voor Angular?'.

Om tot een objectief resultaat te komen, worden beide frameworks vergeleken aan de hand van enkele parameters zoals het werken met data, moeilijkheidsgraad, compatibiliteit voor mobile, populariteit van het framework... \\
De vergelijkingen gebeuren aan de hand van twee gelijkaardige contactapplicaties. Een daarvan is opgebouwd in Angular, de andere in Vaadin. Beide applicaties maken gebruik van een Spring Boot backend. 

Uit dit onderzoek is gebleken dat voor kleine applicaties de keuze bij de programmeur zelf gelegd kan worden. Indien de programmeur een voorkeur heeft voor Java kan hij best opteren voor Vaadin. Indien hij goed thuis is in de webtechnologieën is Angular de logische keuze.

Voor grote applicaties is Angular steeds de betere keuze, dit door de grote community achter dit framework. Dankzij deze community is de continuïteit van dit framework verzekerd. Ook voor mobiele applicaties is Angular een betere optie dan Vaadin. 

Een mogelijk vervolg voor dit onderzoek is om nog andere component gebaseerde frameworks te betrekken in de vergelijking, zoals Vue.js of React. Verder kan dit onderzoek ook opnieuw worden uitgevoerd telkens wanneer een nieuwe release van Vaadin of Angular gebeurt. 

