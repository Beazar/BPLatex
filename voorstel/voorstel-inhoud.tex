%---------- Inleiding ---------------------------------------------------------

\section{Introductie} % The \section*{} command stops section numbering
De keuze uit de bestaande frameworks wordt steeds moeilijker door het groeiende aantal opties. Deze bachelorproef zal een antwoord formuleren op de volgende onderzoeksvraag: in welke situatie kiest men best voor het Vaadin framework en wanneer opteert men best voor Angular? Dit onderzoek zal de keuze tussen twee populaire component based frameworks, Vaadin en Angular, vereenvoudigen. Uit onderzoek van ~\textcite{Schlosser2018} blijkt dat Angular op de tweede plaats komt qua populariteit terwijl Vaadin de achtste plaats bezet. 

%---------- Stand van zaken ---------------------------------------------------

\section{State-of-the-art}
\label{sec:state-of-the-art}

Het belangrijkste verschil tussen de twee frameworks is de programmeertaal waarmee de frameworks werken. Vaadin werkt voornamelijk met Java maar kan ook werken met andere JVM talen zoals Javascript. Angular werkt met HTML, CSS en Typescript of Javascript.  Een tweede belangrijk verschil is dat Angular een frontend framework is in tegenstelling tot Vaadin dat zowel frontend als backend in zijn framework omvat.
Een derde verschil is dat Angular gebruik maakt van een Mobile-first aanpak, wat niet het geval is bij Vaadin.
Ten slotte is Angular beter geschikt voor applicaties die schaalbaar moeten zijn voor een groot aantal gebruikers. Vaadin krijgt moeilijkheden wanneer er meer dan 100.000 gelijktijdige gebruikers zijn. 

In 2017 is er een vergelijkbaar onderzoek uitgevoerd door \textcite{Hellberg2017} waaruit blijkt dat de keuze tussen Vaadin en Angular afhankelijk is van een aantal factoren zoals: kwaliteiten van het team, de verwachtingen omtrent schaalbaarheid en onderhoud, beveiliging en de eisen voor mobile.
In het onderzoek van Hellberg werd gewerkt met Vaadin 8 en Angular 4.

Dit onderzoek zal zich onderscheiden van bovenstaand onderzoek door te werken met nieuwe releases, namelijk Vaadin 10 en Angular 7. 

% Voor literatuurverwijzingen zijn er twee belangrijke commando's:
% \autocite{KEY} => (Auteur, jaartal) Gebruik dit als de naam van de auteur
%   geen onderdeel is van de zin.
% \textcite{KEY} => Auteur (jaartal)  Gebruik dit als de auteursnaam wel een
%   functie heeft in de zin (bv. ``Uit onderzoek door Doll & Hill (1954) bleek
%   ...'')


%---------- Methodologie ------------------------------------------------------
\section{Methodologie}
\label{sec:methodologie}

De frameworks zullen vergelijkingen ondergaan op verschillende vlakken zoals de architectuur en structuur van het framework, populariteit,  het werken met data, mobile ondersteuning, testing, onderhoud en performantie. Deze vergelijking zal gebeuren aan de hand van een simulatie waarbij een applicatie tweemaal wordt opgebouwd. De eerste keer gebeurt dit aan de hand van het Vaadin framework, de tweede keer aan de hand van Angular. 
%---------- Verwachte resultaten ----------------------------------------------
\section{Verwachte resultaten}
\label{sec:verwachte_resultaten}

De verwachte resultaten houden in dat het opzetten van een webapplicatie aan de hand van Vaadin eenvoudiger zal zijn dan aan de hand van Angular. Het aanpassen van de applicatie zal echter eenvoudiger zijn in Angular omdat hier gewerkt wordt met CSS. Dit houdt in dat ook ontwerpers die niet geschoold zijn in het programmeren kunnen meewerken aan de look-and-feel van de applicatie. 
Qua mobile support kan er verwacht worden dat Angular beter zal scoren dan Vaadin, dankzij zijn mobile-first approach.
De performantie van Vaadin zou hoger moeten liggen dan die van Angular in de eerste keer dat de applicatie geladen wordt. De volgende keren zouden de laadtijden gelijkaardig moeten zijn. 
Ten slotte zal Vaadin beter scoren op vlak van beveiliging omdat Angular gebruik maakt van third party UI components, wat de ondersteuning fragiel maakt. 

%---------- Verwachte conclusies ----------------------------------------------
\section{Verwachte conclusies}
\label{sec:verwachte_conclusies}

Naar verwachting zullen beide frameworks hun toepassingen kennen in verschillende situaties. Geen van beide frameworks zal beduidend beter zijn dan het andere, maar in verschillende situaties zal er geopteerd worden voor verschillende frameworks. Voor het starten van een project moet eerst een duidelijke analyse gemaakt worden van de requirements om zo te bepalen welk framework de beste resultaten zal bieden. Parameters zoals daar zijn beveiliging, kwaliteiten van het team en schaalbaarheid zullen een belangrijk onderdeel zijn van de uiteindelijke beslissing. Bovendien zal de scholing van de programmeur vaak doorslaggevend zijn in de eindbeslissing. Een Java programmeur zal zich hoogstwaarschijnlijk comfortabeler voelen met het Vaadin framework, terwijl een Typescript programmeur sneller zal kiezen voor het Angular framework. 

